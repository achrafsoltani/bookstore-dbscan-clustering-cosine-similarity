Recommendation systems are essential in various domains, from e-commerce to streaming services, by predicting user preferences and providing personalized suggestions. However, the effectiveness of such systems greatly relies on the quality of the data representation and the similarity measures used. This paper explores the development of a recommendation system using cosine similarity as the core metric for identifying user-item affinities. We examine key steps in the data and feature engineering process, including the transformation, selection, and normalization of features to build a robust representation space. A book recommendation scenario is employed as a case study to illustrate the proposed methodology. Preliminary design choices and exploratory analyses are presented, laying the groundwork for future experimentation and evaluation.

\textbf{\textit{Keywords:}} Recommendation Systems, Feature Engineering, Cosine Similarity, Book Recommendation, Data Preparation.